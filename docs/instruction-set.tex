\documentclass{article}
\usepackage{amsmath}
\usepackage[utf8]{inputenc}
\usepackage{booktabs}
\usepackage{microtype}
\usepackage{listings}
\usepackage[colorinlistoftodos]{todonotes}
\usepackage{systeme}
\usepackage{graphicx}
\usepackage{float}
\usepackage{blindtext}
\usepackage{needspace}

\lstdefinelanguage{ASM}
{
  morekeywords={ldr, mov},
  sensitive=false,
  morecomment=[l]{;},
  morecomment=[s]{/*}{*/},
}

\lstset{
  language=ASM,
  basicstyle=\ttfamily,
  keywordstyle=\bfseries,
  commentstyle=\itshape,
  breaklines=true,
  frame=single,
  numbers=left,
  numberstyle=\tiny,
  captionpos=b,
  tabsize=8,
  float,
}
\pagestyle{empty}

\title{Instruction Set -inkorrekt \\\small 16 bit emulator}
\author{Gustaf Franzén och Adam Månsson}

\begin{document}
	\maketitle

	\section*{Specifikationer}

	\subsection*{Generella}
	16-bitars little-endian cpu \newline
	32-bitars-address bus \newline
	4-Mb ram från address \(0x00000000 - 0x003FFFFF\) \newline
	4-Gb rom från \(0x003FFFFF - 0xFFFFFFFF\) \newline
	16-bit IO controller

	\subsection*{Register}
	10 16-bitars register \(r0 - r9\) \newline
	10 8-bitars register \(b0 - b9\) som motsvarar de minst
	signifikanta bitarna i motsvarande 16 bitars register
	\newline
	16-bitars Stack pointer register \(sp\) \newline
	16-bitars Status register \(sr\) \newline
	32-bitars program-counter register \(pc\)

	r16 refererar till ett 16-bitars register (r0-r9 | sp), r8 ett 8-bitars

	\subsection*{Addressing-modes}
	processorn stödjer 8 sätt att accessa minne från 32-bitars bussen.
	i Direkt addressering specifiseras en address direkt medans
	i inderekt addressering specificeras en address där den direkta addressen ligger.
	Absolut adressering menas att hela 32-bitars adressen är specificerad
	medans med zeropage addressing specificeras endast en 16-bitars adress, därför kommer man endast åt de första \(2^16\) addresserna
	\newline
	\begin{table}[H]
		\centering
		\begin{tabular}{|c|c|}
		\hline
		\textbf{type} & \textbf{description}\\
		\hline
		r16 & 16-bit register (r0-r9 | sp) \\
		\hline
		r8 & 8-bit register b0-b9 \\
		\hline
		im16 & 16-bit immidiate value \\
		\hline
		im32 & 32-bit immidiate value \\
		\hline
		[\textbf{value}]:8 & the 8-bit stored at \textbf{value} \\
		\hline
		[\textbf{value}]:16 & the 16-bit value stored at \textbf{value} \\
		\hline
		[\textbf{value}]:32 & the 32-bit value stored at \textbf{value} \\
		\hline
		\end{tabular}
		\caption{H:r16, L:r16}
		\label{tab:dir-abs}
	\end{table}
	\subsubsection*{Direct Absolute}
	\begin{table}[H]
		\centering
		\begin{tabular}{|p{2cm}|p{2cm}|p{2cm}|}
		\hline
		\textbf{abs} & \multicolumn{2}{|c|}{direct} \\
		\hline
		\textbf{args} & \(\textless H:r16 \textgreater\) & \(\textless L:r16 \textgreater\) \\
		\hline
		\end{tabular}
	\end{table}
	In Direct Absolute addressing the absolute adress is formed by
	combining two registers to form a 32 bit address
	\newline
	\textbf{syntax exaple to load r2 from address 0xABCD1234:}
	\newline
	\begin{needspace}{\baselineskip}
	\begin{lstlisting}[language=ASM, label={lst:armcode}]
	mov r0, #0xABCD
	mov r1, #0x1234
	ldr r2, r0, r1
	\end{lstlisting}
	\end{needspace}

	\subsubsection*{Direct Absolute Immidiate}
	\begin{table}[H]
		\centering
		\begin{tabular}{|p{2cm}|p{4cm}|}
		\hline
		\textbf{abs} & direct \\
		\hline
		\textbf{args} & \(\textless V:im32 \textgreater\) \\
		\hline
		\end{tabular}
	\end{table}
	In Direct Absolute Immidiate addressing the absolute adress is
	supplied as a 32 bit immidiate value
	\newline
	\textbf{syntax exaple to load r2 from address 0xABCD1234:}
	\newline
	\begin{needspace}{\baselineskip}
	\begin{lstlisting}[language=ASM, label={lst:armcode}]
	ldr r2, #0xABCD1234
	\end{lstlisting}
	\end{needspace}

	\subsubsection*{Direct Zeropage}
	\begin{table}[H]
		\centering
		\begin{tabular}{|p{2cm}|p{2cm}|p{2cm}|}
		\hline
		\textbf{abs} & \multicolumn{2}{|c|}{direct} \\
		\hline
		\textbf{args} & \(\textless 0x0000 \textgreater\) & \(\textless Z:r16 \textgreater\) \\
		\hline
		\end{tabular}
	\end{table}
	Direct Zeropage addressing the absolute adress is
	suplied as a 16-bit register and padded with 0s
	\newline
	\textbf{syntax exaple to load r1 from address 0x1234:}
	\newline
	\begin{needspace}{\baselineskip}
	\begin{lstlisting}[language=ASM, label={lst:armcode}]
	mov r0, #1234
	ldr r1, r0
	\end{lstlisting}
	\end{needspace}

	\subsubsection*{Direct Zeropage Immidiate-offset}
	\begin{table}[H]
		\centering
		\begin{tabular}{|p{2cm}|p{2cm}|p{2cm}|}
		\hline
		\textbf{abs} & \multicolumn{2}{|c|}{direct} \\
		\hline
		\textbf{args} & \(\textless 0x0000 \textgreater\) & \(\textless R:r16 \textgreater + \textless O:im16 \textgreater\) \\
		\hline
		\end{tabular}
	\end{table}
	Direct Zeropage Immidiate-offset addressing the absolute adress is
	formed by a 16-bit register added with a immidiate offset, and padded with 0s
	\newline
	\textbf{syntax exaple to load r1 from address 0x1235:}
	\newline
	\begin{needspace}{\baselineskip}
	\begin{lstlisting}[language=ASM, label={lst:armcode}]
	mov r0, #1234
	ldr r1, r0, #1
	\end{lstlisting}
	\end{needspace}

	\subsubsection*{Inderect Absolute Immidiate}
	\begin{table}[H]
		\centering
		\begin{tabular}{|p{2cm}|p{4cm}|}
		\hline
		\textbf{abs} & direct=[inderect]:32 \\
		\hline
		\textbf{abs} & inderect \\
		\hline
		\textbf{args} & \(\textless V :im32\textgreater\) \\
		\hline
		\end{tabular}
		\caption{V:im32}
		\label{tab:dir-zp-imof}
	\end{table}
	Inderect Absolute Immidiate addressing the absolute adress is
	read from the supplied address
	\newline
	\textbf{syntax exaple to load r1 with 10 from address(0x00FF0000) stored at 0x1235}
	\newline
	\begin{needspace}{\baselineskip}
	\begin{lstlisting}[language=ASM, label={lst:armcode}]
	mov r0, #10
	str r0, #0x00FF0000

	mov r0, #0x0000
	mov r1, #0x00FF

	str r0, #0x00001235
	str r1, #0x00001235 + #1 ; store in little endian

	ldr r1, [#0x00001235]
	\end{lstlisting}
	\end{needspace}















\end{document}
